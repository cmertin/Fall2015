\section{Mistake-bound learning}
\label{sec:mistake-bound-learning}

\begin{enumerate}
\item Let us define a concept class $C$ of Boolean functions from
  $\{0,1\}^n$ to $\{0, 1\}$. Every function in this class is an
  indicator function for a particular element in the input space. That
  is, for every $\bz \in \{0, 1\}^n$, there is a function $f_\bz$ in $C$
  defined as follows:

  \begin{equation}
    f_\bz(\bx) = \begin{cases}
      1 & \mbox{if } \bz = \bx \\
      0 & \mbox{otherwise }
    \end{cases}
  \end{equation}

  \begin{enumerate}
  \item ~[2 points] How many functions does $C$ contain?
    \begin{itemize}
      \item $2^{n}$
    \end{itemize}
  \item ~[10 points] Suppose we use the Halving algorithm to learn this
    concept class. How many mistakes will the algorithm make? Write a
    short proof for your answer. (Hint: The bound we proved in class
    doesn't directly apply.)
    \begin{align}
      \intertext{$2^{n}$ possible combinations}
      1z &\leq \frac{1}{2}|C_{n-1}|\\
         &\leq \frac{1}{2}\cdot\frac{1}{2}|C_{n-2}|\\
         &\vdots\\
         &\leq \frac{1}{2^{n}}\cdot\frac{1}{2^{n}} = \frac{1}{2^{n}}^{n} \geq 1\\
         &\Rightarrow \mathcal{O}(n\log(n))
    \end{align}
  \item ~[1 point] Is Halving a mistake-bound algorithm for this
    concept class?
    \begin{itemize}
      \item No
    \end{itemize}
  \end{enumerate}

\item ~[12 points] Recall from class that the Halving algorithm
  assumes that there is the true (hidden) function is in the concept
  class $C$ with $N$ elements and tries to find it. In this setting,
  we know the number of mistakes made by the algorithm is $O(\log N)$.

  Another way to think about this setting is that we are trying to
  predict with expert advice. That is, we have a pool of $N$ so called
  experts, only one of whom is perfect. As the halving algorithm
  proceeds, it cuts down this pool by at least half each time a
  mistake is made.

  Suppose, instead of one perfect expert, we have $M$ perfect experts
  in our pool. Show that the mistake bound of the same Halving
  algorithm in this case is $O(\log \frac{N}{M})$.
  
  ({\em Hint}: To show this, consider the stopping condition of the
  algorithm. At what point, will the algorithm stop making mistakes?)
  \begin{align}
    \intertext{Would only make $\frac{N}{M}$ mistakes}
    1 &= |C_{n}| < \frac{1}{2}|C_{n-1}|\\
      &< \frac{1}{2}\cdot \frac{1}{2} |C_{n-2}|\\
      &< \vdots\\
      &< \frac{1}{2^{n}}|C_{n}| = \frac{1}{2^{n}}\frac{N}{M}\\
      \intertext{Which comes about because the base case $C_{0}$ would be reacing the expert in $\frac{N}{M}$ tests}
      2^{n} &\leq \frac{N}{M}\\
      &\log\left(\frac{N}{M}\right)
  \end{align}
\end{enumerate}

%%% Local Variables:
%%% mode: latex
%%% TeX-master: "hw2"
%%% End:
