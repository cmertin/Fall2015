\section{Na\"ive Bayes}
\label{sec:bayes}
Consider the Boolean function $f_{TH(3,7)}$. This is a threshold
function defined on the 7 dimensional Boolean cube as follows: given
an instance $x$, $f_{TH(3,7)}(x) = 1$ if and only if 3 or more of
$x$'s components are 1.

\begin{enumerate}
\item ~[4 points] Show that $f_{TH(3,7)}$ has a linear decision surface over the 7 dimensional Boolean cube.

\begin{itemize}
\item The weight vector can be defined as such $\mathbf{w}=\left(1,1,1,1,1,1,1\right)^{T}$, with the bias, $\theta$, can be defined such that $\theta = -3$. This will give the result that if $\mathbf{w}^{T}\mathbf{x}_{i}+\theta\geq 0$, then $y=1$, otherwise $y=-1$. This will result in the bounds such that if at least $3$ of the components are $1$, then the result would be $y=1$.
\end{itemize}

\item ~[7 points] Assume that you are given data sampled according to the uniform
  distribution over the Boolean cube $\{0, 1\}^7$ and labeled
  according to $f_{TH(3,7)}$. Use na\"ive Bayes to learn a hypothesis
  that predicts these labels. What is the hypothesis generated by the
  na\"ive Bayes algorithm? (You do not have to implement the algorithm
  here. You may assume that you have seen all the data required to get
  accurate estimates of the probabilities).

\begin{itemize}
\item Let $x_{i}$ be the $i^{th}$ input of the data in the cube for $i = \{1,2,\ldots,n\}$. We can find the most probable hypothesis by
\begin{align}
p(h|\mathcal{D}) &= \text{arg}\max_{h\in H}p\left(y_{i}|x_{i}\right)\\
\intertext{The question states that we are sampling from a normal distribution over the cube, we can represent all the data in the following way}
 &\Rightarrow \text{arg}\max_{y\in \{0,1\}}p(y)\prod_{i=1}^{7}p\left(x_{i}|y\right)\\
\intertext{which expands into the form}
\max &\left[ p(0)\prod_{i=1}^{7}p\left(x_{i}|0 \right), p(1)\prod_{i=1}^{7}p\left(x_{i}|1 \right)\right]\\
p(0) &= \text{probability that at least 5 elements are 0}\\
&= \left(\frac{1}{2}\right)^{7}\sum_{i=5}^{7}{7 \choose i} = \frac{29}{128}\\
p(1) &= 1 - p(0) = \frac{99}{128}
\intertext{We now need to define $p(x_{i}=0|y)$ and $p(x_{i}=1|y)$, for $y = \{ 0, 1\}$. These would be the same for all values of $i$ since they come from the same distribution}
p(x_{i}|0) &= \frac{p(x_{i})p(0|x_{i})}{p(0)}\label{eq:all}\\
\intertext{For $x_{i}=1$, we need to calculate the probability of $p(0|x_{i}=1)$, which is the probability of {\em at least} 5 of the remaining 6 $x_{i}$'s being 0}
p(0|x_{i}=1) &= \left( \frac{1}{2}\right)^{6}\left[ {6 \choose 5} + {6 \choose 6}\right] = \frac{7}{64}\\
\intertext{Now we can calculate $p(x_{i}=1|0)$ from Equation (\ref{eq:all})}
p(x_{i}=1|0) &= \frac{\frac{1}{2}\frac{7}{64}}{\frac{29}{128}} = \frac{7}{29}
\intertext{Where we can do the same thing for $x_{i}=0$, which would be the probability of at least 4 of the remaining 6 being 0}
p(0|x_{i}=0) &= \left(\frac{1}{2}\right)^{6}\left[ \sum_{i=4}^{6}{6 \choose i}\right] = \frac{22}{64}
\intertext{where from here we can calculate $p(x_{i}=0|0)$ from Equation (\ref{eq:all}) again as being}
p(x_{i}=0|0) &= \frac{\frac{1}{2}\frac{22}{64}}{\frac{29}{128}} = \frac{22}{29}
\intertext{Finally, we need to calculate it for $y=1$, whcih can be done as follows}
p(1|x_{i}=1) &= 1 - p(0|x_{i}=1) = 1-\frac{7}{64} = \frac{57}{64}\\
p(1|x_{i}=0) &= 1 - p(0|x_{i}=0) = 1-\frac{22}{64} = \frac{42}{64}\\
\intertext{Leading to the final results of}
p(x_{i}=1|1) &= \frac{\frac{1}{2}\frac{57}{64}}{\frac{99}{128}} = \frac{57}{99}\\
p(x_{i}=0|1) &= \frac{\frac{1}{2}\frac{42}{64}}{\frac{99}{128}} = \frac{42}{99}\\
\intertext{From these we can build the hypothesis}
h(\mathbf{x}) &= \text{arg}\max_{y\in \{0,1\}} \frac{29 + 70y}{128}\prod_{i=1}^{7}\frac{42 + 15\mathbf{x}_{i}}{99}y - \frac{22 - 15\mathbf{x}_{i}}{29}(y-1)\label{eq:hyp}
\end{align}
\end{itemize}

\item ~[4 points] Show that the hypothesis produced in the previous question does not represent this function.
\begin{itemize}
\item This can be proven via {\em proof by contradiction}. Assume we have the vector $\mathbf{x}=(0,0,0,1,0,0,1)^{T}$ and that the hypothesis in the answer to the previous question correctly classifies the data. From the first question, we know that this should be classified as $y=0$. We can test this to verify
\begin{align}
h(\mathbf{x}) &= \max_{y\in \{0,1\}}\left\{ \begin{array}{l l r}
\frac{29}{128}\left[ \left(\frac{22}{29}\right)^{5}\left(\frac{7}{29}\right)^{2} \right]\approx & 0.003317 & (y=0)\\
\frac{99}{128}\left[ \left(\frac{57}{99}\right)^{2}\left(\frac{42}{99}\right)^{5} \right]\approx & 0.003524 & (y=1)
\end{array}\right.
\end{align}
which the hypothesis would choose the largest number which would be $y=1$, which isn't true as it should be $y=0$. This is a contradiction, therefore it doesn't describe this function.
\end{itemize}

\item ~[5 points] Are the na\"ive Bayes assumptions satisfied by
  $f_{TH(3,7)}$? Justify your answer.

\begin{itemize}
\item No, the na\"{i}ve bayes assumptions are not satisfied by $f_{TH(3,7)}$. The assumption that is made is that $\mathbf{x}_{i}$ is independent for each value of $i$, which isn't actually the case. Any of the probability calculations should include the probability of the other features as well.
\end{itemize}

\end{enumerate}
%%% Local Variables:
%%% mode: latex
%%% TeX-master: "hw5"
%%% End:
